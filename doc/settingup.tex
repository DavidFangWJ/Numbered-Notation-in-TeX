\chapter{输入文档的格式}
\section{\musixtex 的基本规定}
\musixtex 的输入文件和一般的\TeX 文件类似。使用\musixtex 宏包的命令是\verb|\input musixtex|;相应地,使用简谱宏包\musixjnp 的命令是\verb|\input musixjnp|。宏集内的其他宏包也如此调用。

\musixtex\enspace 宏集默认乐谱最多有6个声部。如果需要更多的声部,可以使用\textsf{musixadd}\enspace 或\enspace\textsf{musixmad}\enspace 宏包扩展到9或12个。如果还需要更多声部,可以使用\penalty-10000\verb|\setmaxinstruments|、\hfill\verb|\setmaxgroups|、\hfill\verb|\setmaxslurs|、\hfill\verb|\setmaxtrills|、\penalty-10000\verb|\setmaxoctlines|等命令手动修改。


\section{常用的设置命令}\label{whatspecify}

\subsection{全局乐谱尺寸}
\musixtex 设置了4种乐谱尺寸:小(16 pt)、中(20 pt,默认)、大(24 pt)、特大(29 pt)。或需要非默认尺寸,分别使用命令\ctrlseq{smallmusicsize}、\ctrlseq{largemusicsize}、\ctrlseq{Largemusicsize}。在使用该命令时,所有的音乐内容都会相应缩放。

简谱推荐使用“小”尺寸;线简混排时参考第??章的内容。

\subsection{乐器个数}
命令\ctrlseq{instrumentnumber} $n$表示乐谱中乐器的个数,默认为1.

\subsection{谱号}
各谱表的默认谱号是高音谱号。修改谱号的命令为\ctrlseq{setclef}\verb|{|$n$\verb|}{|$s_1s_2s_3s_4\ldots$\verb|}|。$n$是乐器从下往上数的序号,$s_1$、$s_2$等是该乐器从下往上的谱表序号对应的谱号。

简谱应当使用8(没有谱号)。

\subsection{节拍}
所有声部通用的节拍记号可以使用\ctrlseq{generalmeter}\verb|{|$m$\verb|}|命令。$m$是拍号相应的命令,简谱使用\ctrlseq{inijfrac}\verb|{|$a$\verb|}{|$b$\verb|}|(乐谱首)或\ctrlseq{jfrac}\verb|{|$a$\verb|}{|$b$\verb|}|(乐谱中切换拍号),a为上方的数,b为下方的数。

若使某乐器的节拍与全局设置不同,则应该使用
\ctrlseq{setmeter}\verb|{|$n$\verb|}{{|$m_1$\verb|}{|$m_2$\verb|}|\penalty-10000 \verb|{|$m_3$\verb|}{|$m_4$\verb|}}|,与谱表相似。

\subsection{乐器名}
乐器名使用\ctrlseq{setname}\verb|{|$n$\verb|}{|{\fangsong 乐器名}\verb|}|命令设置,程序会自动将乐器名称居中放置在\ctrlseq{parindent}宽度的范围内。

\subsection{乐器组}\label{curlybrackets}
默认情况下,乐器间没有连接。若需要排合唱谱(如上一章中的《喀秋莎》例谱),则需要方括号分组。插入方括号的命令是
\begin{quote}
\ctrlseq{songtop}\verb|{|$n$\verb|}|\\
\ctrlseq{songbottom}\verb|{|$m$\verb|}|
\end{quote}

若需要多个方括号,则使用类似的\ctrlseq{grouptop}和\ctrlseq{groupbottom}命令。

\section{纯简谱谱表常用的设置}
除上述内容以外,简谱谱表还经常需要使用\ctrlseq{nostartrules}(取消每行左侧的连线)和\ctrlseq{sepbarrules}(每个乐器的小节线分开)两条命令。


\endinput


