\vskip 60pt
\begin{center}%
    {\Huge\bfseries\textsf{musixjnp}宏包\\[\bigskipamount]
        \LARGE \TeX{}排版系统下的简谱解决方案\\
        \Large\textit{\textbeta-1.1 版本} \par}%
    \vskip 3em%
    {\large
        \lineskip .75em%
        \begin{tabular}[t]{c}%
            \Large 深圳外国语学校高三(21)班\quad 方惟佳
        \end{tabular}\par}%
    \vskip 1.5em%
    {\large \today 修订 \par}%       % Set date in \large size.
\end{center}\par
\vfill
\thispagestyle{empty}

{\setlength{\leftskip}{7.5cm}
\fangsong
对于不熟悉\TeX 的人,我推荐使用其他软件进行音乐排版。
在电脑上设置\TeX 和MusiX\TeX
并熟练使用这两款软件是一项大工程,
需要耗费很多时间和磁盘空间。

但是,一旦掌握了的话……\\
\hfill\hbox to 0pt{\hss——汉斯·古伊根斯,约1995年}\\

\vspace*{4ex}
我认为,上面这句话过时了。\\
\hfill\hbox to 0pt{\hss——克里斯多弗·比布里彻,2006年}\\

\setlength{\leftskip}{0cm}}

\clearpage

\pagenumbering{roman}\setcounter{page}{2}


\vspace*{20ex}
\begin{quote}
    \setlength{\parindent}{2zw}\setlength{\parskip}{0pt}
    
    \hspace{2zw}\fangsong 本宏包基于GPL协议(最新版本)公开,可以照协议自由复制、使用。任何人可以将其全部或部分代码任意使用,
    但是不能将其自己的产品称作\textsf{musixjnp},除非是修正程序的漏洞(如本版本仍未解决的
    \texttt{\char92 ifnum}嵌套问题)。
    
    对于某些实现(如字体、减时线连接机制)的改动需要以单独的\TeX 或\LaTeX 文档实现。
\end{quote}

\clearpage

\chapter*{前言}
\addcontentsline{toc}{chapter}{前言}
\textsf{musixjnp}是本人在MusiX\TeX 宏集的基础上实现的简谱排版系统,在开发过程中也参考了
\textsf{musixgre}和\textsf{musixper}的代码。在编写过程中,QQ群上的几位朋友也对我的工作在字体等方面给予了相应的帮助。目前\textbeta-1.1版本的最后一个大更新是平连线的制作。

在\TeX 下,五线谱排版已经有\textbf{PMX}(器乐)、\mbox{\textbf{M-Tx}}(声乐)等预处理器。它们的存在简化了用户的学习量和输入。但是,对于排版简谱或线—简混排乐谱,目前仍旧需要学习\textsf{musixjnp}宏包及MusiX\TeX 宏集的相关语法。

适用于简谱的预处理器目前也正在制作中;在预处理器完成之后,用户只需学习预处理器下的简化语法,多数情况下已不再需要了解MusiX\TeX 宏集的相应语法。

MusiX\TeX、\textbf{PMX}、\mbox{\textbf{M-Tx}}等的输入可以通过任意的文本编辑器完成。目前,本宏包(包括MusiX\TeX 宏集中的其他包)不存在可视化的输入界面。

\textsf{注意}:本文还对\musixtex 宏集说明书中与简谱相关的部分做了较为简略的翻译;只与五线谱相关的部分则略去。

%The \href{http://icking-music-archive.org/software/indexmt6.html}
%{\underline{Werner Icking Music Archive}}\ (WIMA) contains excellent and detailed
%instructions for installing \TeX, \musixtex{} and the strongly recommended
%preprocessors \textbf{PMX}
%and \mbox{\textbf{M-Tx}} on
%\href{http://icking-music-archive.org/software/htdocs/Getting_Started_Four_Scenar.html#SECTION00022000000000000000}
%{\underline{Linux/\unix}}, 
%\href{http://icking-music-archive.org/software/htdocs/Getting_Started_Four_Scenar.html#SECTION00021000000000000000}
%{\underline{Windows}} and 
%\href{http://icking-music-archive.org/software/htdocs/Getting_Started_Four_Scenar.html#SECTION00023000000000000000}
%{\underline{Mac OS}}. 
%See
%\href{http://icking-music-archive.org/software/htdocs/Introduction.html#SECTION00012000000000000000}
%{\underline{this}}
%page at WIMA for documentation of 
%\textbf{PMX} and \mbox{\textbf{M-Tx}}.

\begin{flushright}
    方惟佳\\ \today
\end{flushright}

\clearpage

\chapter*{目录}
\addcontentsline{toc}{chapter}{目录}
\makeatletter
\@starttoc{toc}
\makeatother

\clearpage
\setcounter{page}{1}
\pagenumbering{arabic}
\renewcommand{\thepage}{\arabic{page}}