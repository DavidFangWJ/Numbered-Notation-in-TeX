\documentclass[a4paper,zihao=-4, twoside]{ctexrep}
\usepackage[hmargin=3cm,vmargin=2.5cm]{geometry}
\usepackage{textgreek}
\usepackage{lmodern}
\usepackage{siunitx}
\usepackage{musixtex}
\usepackage{musixjnp}

\DeclareFontFamily{JY2}{zhsong}{}
\DeclareFontFamily{JY2}{zhhei}{}
\DeclareFontFamily{JY2}{zhfs}{}
\DeclareFontFamily{JY2}{zhkai}{}

\DeclareFontShape{JY2}{zhsong}{m}{n}{<->s*[0.962216]nshusong-h}{}
\DeclareFontShape{JY2}{zhsong}{bx}{n}{<->s*[0.962216]xiaobiaosong}{}
\DeclareFontShape{JY2}{zhsong}{m}{it}{<->s*[0.962216]kaiti}{}
\DeclareFontShape{JY2}{zhsong}{m}{sl}{<->s*[0.962216]kaiti}{}
\DeclareFontShape{JY2}{zhhei}{m}{n}{<->s*[0.962216]heiti}{}
\DeclareFontShape{JY2}{zhhei}{bx}{n}{<->s*[0.962216]dahei}{}
\DeclareFontShape{JY2}{zhfs}{m}{n}{<->s*[0.962216]fangsong}{}

\AtBeginDvi{
    \special{pdf:mapline fzxss-h UniGB-UTF16-H  FZXSSK.TTF}
    \special{pdf:mapline fzss-h  UniGB-UTF16-H  FZSSK.TTF}
    \special{pdf:mapline syst4-h UniGB-UTF16-H  FZSSK.TTF}
    \special{pdf:mapline fzxbs  UniGB-UTF16-H  FZXBSK.TTF}
    \special{pdf:mapline fzht   UniGB-UTF16-H  FZHTK.TTF}
    \special{pdf:mapline fzdh   UniGB-UTF16-H  fzdhtk.ttf}
    \special{pdf:mapline fzfs   UniGB-UTF16-H  FZFSK.TTF}
    \special{pdf:mapline fzkt   UniGB-UTF16-H  FZKTK.TTF}
}

\renewcommand{\CJKrmdefault}{zhsong}
\renewcommand{\CJKsfdefault}{zhhei}
\renewcommand{\CJKttdefault}{zhfs}
%\renewcommand{\heiti}{\CJKfamily{zhhei}\bfseries}

\AtBeginDvi{
	\special{pdf:mapline fzss   UniGB-UTF16-H  fzssk.ttf}
	\special{pdf:mapline fzht   UniGB-UTF16-H  fzhtk.ttf}
	\special{pdf:mapline fzfs   UniGB-UTF16-H  fzfsk.ttf}
	\special{pdf:mapline fzxbs  UniGB-UTF16-H  fzxbsk.ttf}
}

%\西暦
\title{\textsf{musixjnp}\\——试验性的\TeX 简谱宏包}

\newcommand{\ctrlseq}[1]{\hspace{\xkanjiskip}\texttt{\char92 #1}}
\newcommand{\zctrlseq}[1]{\texttt{\char92 #1}}
\newcommand{\musixjnp}{\textsf{musixjnp}}

\begin{document}

\vskip 60pt
\begin{center}%
    {\Huge\bfseries\textsf{musixjnp}宏包\\[\bigskipamount]
        \LARGE \TeX{}排版系统下的简谱解决方案\\
        \Large\textit{\textbeta-1.1 版本} \par}%
    \vskip 3em%
    {\large
        \lineskip .75em%
        \begin{tabular}[t]{c}%
            \Large 深圳外国语学校高三(21)班\quad 方惟佳
        \end{tabular}\par}%
    \vskip 1.5em%
    {\large \today 修订 \par}%       % Set date in \large size.
\end{center}\par
\vfill
\thispagestyle{empty}

{\setlength{\leftskip}{7.5cm}
\fangsong
对于不熟悉\TeX 的人,我推荐使用其他软件进行音乐排版。
在电脑上设置\TeX 和MusiX\TeX
并熟练使用这两款软件是一项大工程,
需要耗费很多时间和磁盘空间。

但是,一旦掌握了的话……\\
\hfill\hbox to 0pt{\hss——汉斯·古伊根斯,约1995年}\\

\vspace*{4ex}
我认为,上面这句话过时了。\\
\hfill\hbox to 0pt{\hss——克里斯多弗·比布里彻,2006年}\\

\setlength{\leftskip}{0cm}}

\clearpage

\pagenumbering{roman}\setcounter{page}{2}


\vspace*{20ex}
\begin{quote}
    \setlength{\parindent}{2zw}\setlength{\parskip}{0pt}
    
    \hspace{2zw}\fangsong 本宏包基于GPL协议(最新版本)公开,可以照协议自由复制、使用。任何人可以将其全部或部分代码任意使用,
    但是不能将其自己的产品称作\textsf{musixjnp},除非是修正程序的漏洞(如本版本仍未解决的
    \texttt{\char92 ifnum}嵌套问题)。
    
    对于某些实现(如字体、减时线连接机制)的改动需要以单独的\TeX 或\LaTeX 文档实现。
\end{quote}

\clearpage

\chapter*{前言}
\addcontentsline{toc}{chapter}{前言}
\textsf{musixjnp}是本人在MusiX\TeX 宏集的基础上实现的简谱排版系统,在开发过程中也参考了
\textsf{musixgre}和\textsf{musixper}的代码。在编写过程中,QQ群上的几位朋友也对我的工作在字体等方面给予了相应的帮助。目前\textbeta-1.1版本的最后一个大更新是平连线的制作。

在\TeX 下,五线谱排版已经有\textbf{PMX}(器乐)、\mbox{\textbf{M-Tx}}(声乐)等预处理器。它们的存在简化了用户的学习量和输入。但是,对于排版简谱或线—简混排乐谱,目前仍旧需要学习\textsf{musixjnp}宏包及MusiX\TeX 宏集的相关语法。

适用于简谱的预处理器目前也正在制作中;在预处理器完成之后,用户只需学习预处理器下的简化语法,多数情况下已不再需要了解MusiX\TeX 宏集的相应语法。

MusiX\TeX、\textbf{PMX}、\mbox{\textbf{M-Tx}}等的输入可以通过任意的文本编辑器完成。目前,本宏包(包括MusiX\TeX 宏集中的其他包)不存在可视化的输入界面。

\textsf{注意}:本文还对\musixtex 宏集说明书中与简谱相关的部分做了较为简略的翻译;只与五线谱相关的部分则略去。

%The \href{http://icking-music-archive.org/software/indexmt6.html}
%{\underline{Werner Icking Music Archive}}\ (WIMA) contains excellent and detailed
%instructions for installing \TeX, \musixtex{} and the strongly recommended
%preprocessors \textbf{PMX}
%and \mbox{\textbf{M-Tx}} on
%\href{http://icking-music-archive.org/software/htdocs/Getting_Started_Four_Scenar.html#SECTION00022000000000000000}
%{\underline{Linux/\unix}}, 
%\href{http://icking-music-archive.org/software/htdocs/Getting_Started_Four_Scenar.html#SECTION00021000000000000000}
%{\underline{Windows}} and 
%\href{http://icking-music-archive.org/software/htdocs/Getting_Started_Four_Scenar.html#SECTION00023000000000000000}
%{\underline{Mac OS}}. 
%See
%\href{http://icking-music-archive.org/software/htdocs/Introduction.html#SECTION00012000000000000000}
%{\underline{this}}
%page at WIMA for documentation of 
%\textbf{PMX} and \mbox{\textbf{M-Tx}}.

\begin{flushright}
    方惟佳\\ \today
\end{flushright}

\clearpage

\chapter*{目录}
\addcontentsline{toc}{chapter}{目录}
\makeatletter
\@starttoc{toc}
\makeatother

\clearpage
\setcounter{page}{1}
\pagenumbering{arabic}
\renewcommand{\thepage}{\arabic{page}}

\setcounter{page}{1}
\pagenumbering{arabic}
\renewcommand{\thepage}{\arabic{page}}

\chapter{\musixtex 及\textsf{musixjnp}简介}
本章不是\musixtex 宏集或\musixjnp 宏包的教程,而是对二者能力的概览。

\musixtex 是一系列宏包及字体的集合,用于在\TeX 系统下进行音乐排版。使用该系统时,电脑上需要有正常工作的\TeX 发行版。
\textsf{musixjnp}是我在\musixtex 及相关宏包上的扩展,目前版本(\textbeta-1.1)已经能够完成较简单的简谱乐曲及简线混排乐曲。

\musixjnp 宏包正如同字模,有音符数字、连线、附加符号等的字形,可以用于排版质量合格的简谱谱面。用户需要通过输入相应的命令,手动指定每个“逻辑符号”的位置。在预处理器完成之后,也可以使用与水平位置无关的命令简化输入。在每个“逻辑符号”(如简谱音符)中,每个字符(如上下加点、连线)的位置均由程序自己计算,足以满足一般需求。

对于歌词,目前可以使用插入文本的方式输入。\verb|musixlyr|宏包目前暂不支持简谱,将来可以通过重写命令的方式增加支持。

本宏包是依据plain \TeX 开发的,但是在\LaTeX 上应该也能正常使用。一般来说,纯乐谱的排版并不需要\LaTeX;\LaTeX 只需要在需要插入简谱示例的文档(如本说明)中使用。
将来针对\LaTeX 进行的最大工程将会是使用NFSS机制进行“文本中插入简谱音符本身”一类的需求,及编写适当的\hspace{\xkanjiskip}\texttt{.sty}文件。

\section{音乐排版的特点}
\subsection{音乐排版是二维的}
除笛子、人声等的乐谱外,多数乐谱都以二维形式存在:横向是时间(节拍、节奏),在同一时刻内可以竖向排列多个音符,称为“簇”。与此同时,文件本身却是一维的。因此,较有逻辑的音符排列方法正如图\ref{readtable}所示。
\begin{figure}
    \begin{center}
        \small
        \renewcommand{\arraystretch}{1.4}
        \setlength{\arrayrulewidth}{1.5pt}
        \setlength{\tabcolsep}{2ex}
        \begin{tabular}{|ll|ll|}
            \fbox{3}&\fbox{6}&\fbox{9}&\fbox{12}\\
            \fbox{2}&\fbox{5}&\fbox{8}&\fbox{11}\\
            \fbox{1}&\fbox{4}&\fbox{7}&\fbox{10}\\
        \end{tabular}
    \end{center}
    \caption{音符在一维文件中的逻辑排列方法}
    \label{readtable}
\end{figure}
因此,\musixtex 宏集中将一个簇(上图中的一列)记录为
\begin{center}
    \verb|\notes ... & ... & ... \en|%
    \footnote{命令\ctrlseq{en}是\ctrlseq{enotes}的缩写,两者等价。}
\end{center}
形式,其中\hspace{\xkanjiskip}\verb|&|\hspace{\xkanjiskip}用于分割自下而上排列的乐器(或声部)。当一个乐器需要使用多个声部时,各个声部使用\hspace{\xkanjiskip}\verb=|=\hspace{\xkanjiskip}字符分隔。

因此,若输入独唱配钢琴伴奏的乐谱,则应使用
\begin{center}
    \verb=\notes ... | ... & ...\en=
\end{center}
表示每一簇。

每一个簇中,一行谱表上不仅可以放置同时出现的和弦音符,也可以放置较短的连续音符。因此,这表示\musixtex 宏集中需要有两种命令,\textsf{占位}命令和不占位命令。后一类命令可以用于表示和弦、演奏记号等。

\subsection{水平间距}
计算乐谱的水平间距十分复杂,不在此文中讨论。水平间距的选择主要是通过\ref{newspacings}节描述的内容完成的。通过改变上文中\ctrlseq{notes}对应的命令,可以指定相对水平间距。

\subsection{输入内容}
\musixjnp 宏包通过类比\musixtex 宏集相应的代码,将输入分为以下几类。
\begin{itemize}\setlength{\itemsep}{0ex}
    \item 简谱音符(增时线视作一种特殊的音符)
    \item 减时线
    \item 简谱连音线(分为弯连线、平连线)的开始、结束
    \item 延音线、连音线的开始、结束
    \item 升降号
    \item 演奏记号
    \item 小节线
    \item 拍号、调号等
\end{itemize}

在五线谱中,字母表示绝对音高。例如,命令\ctrlseq{wh a}表示高\SI{220}{\hertz}的全音符,
\ctrlseq{wh h}高一个八度等。在简谱中,字母\verb|c|则只表示中音do。

五线谱中的和弦使用\ctrlseq{zq}表示无符干的音符进行组合。%例如,以下的C大和弦
%
%\begin{music}\nostartrule
%\startextract\NOtes\zq{ceg}\qu j\en\zendextract
%\end{music}
%\noindent 使用\hspace{\xkanjiskip}\verb|\zq c\zq e\zq g\qu j|或\hspace{\xkanjiskip}
%\verb|\zq{ceg}\qu j|;\verb|u|表示符头向上。
简谱目前暂不支持和弦。

\subsection{符杠及减时线}
这两者都是使用一对命令产生的。第一个命令指明了符杠或减时线的数目和序号。对于符杠来说,还指定了横向位置(默认为当前)、高度、方位(指上方或下方)、斜度。第二个命令指名了结束位置和序号。引入序号的目的是使得符杠或减时线可以重叠。

\subsection{输入文本}
\ctrlseq{zcharnote}命令可以在谱面的任何横、纵位置输入任何符号组合。这项功能允许用户将自定义符号任意地加入到乐谱中。

\section{乐谱示例}
下例是《喀秋莎》选段。

\begin{music}
    \smallmusicsize
    \sepbarrules
    \nobarnumbers
    \instrumentnumber{2}
    \setclef18
    \setlines10
    \setclef28
    \setlines20
    \akkoladen{{1}{2}}
    \startextract
    \NOtes\jq{efe}&\jq{ehg}\en
    \Notes\Djul fe&\Djul hg\en\bar
    \NOtes\jq d&\jq f\en
    \Notes\Djul cb&\Djul ed\en
    \NOtes\jq{ca}&\jq{ea}\en\bar
    \Notes\jc R&\jc R\en
    \NOtes\jq d&\jq f\en
    \Notes\jc b&\jc d\en
    \NOtesp\jqp c&\jqp e\en
    \Notes\jc c&\jc c\en\bar
    \Notes\Djul bL\Djul cb&\Djul bL\Djul cb\en
    \NOtes\jq{aZ}&\jq{aZ}\en
    \endextract
\end{music}

\noindent 相应的代码如下所示。
\begin{quote}
    \fontsize{10pt}{10pt}\selectfont
    \begin{verbatim}
    \smallmusicsize % 简谱一般使用较小的字号
    \sepbarrules % 简谱小节线不连通
    \nobarnumbers % 无小节计数
    \instrumentnumber{2}
    % 两个乐器(声部)
    \setclef18
    % 下方声部无五线谱谱号
    \setlines10 % 无谱线
    \setclef28 % 上方声部相同
    \setlines20
    \akkoladen{{1}{2}} % 方括号连接
    \startextract % 乐谱开始
    \NOtes\jq{efe}&\jq{ehg}\en
    \Notes\Djul fe&\Djul hg\en\bar
    \NOtes\jq d&\jq f\en
    \Notes\Djul cb&\Djul ed\en
    \NOtes\jq{ca}&\jq{ea}\en\bar
    \Notes\jc R&\jc R\en
    \NOtes\jq d&\jq f\en
    \Notes\jc b&\jc d\en
    \NOtesp\jqp c&\jqp e\en
    \Notes\jc c&\jc c\en\bar
    \Notes\Djul bL\Djul cb&\Djul bL\Djul cb\en
    \NOtes\jq{aZ}&\jq{aZ}\en
    \endextract % 乐谱结束
    \end{verbatim}\end{quote}

\section{三段系统}\label{threepass}
\subsection{简介}
\TeX 默认的断行算法适合于文本,因为每行西文文本都会有足够多的空格,无须过大地调整就可以实现两端对齐。但是,五线谱、简谱往往一行不超过5小节,因此小节内部也需要做相应调整——这便超出了\TeX 的处理能力。为解决五线谱两端对齐的计算,\musixtex 原作者发明了一套三段系统。

首先,\musixtex 系统将水平距离分为两大类:\textsf{不可变}和\textsf{可变}。小节线、谱表等前后的距离是不可变的,而音符、休止符前后的间距是可成比例伸缩的。在这个意义上,可变的间距可以定义为基本间距(记为\ctrlseq{elemskip})的倍数。例如,\textbf{PMX}中所有的十六分音符一般宽\verb|1.41\elemskip|。

\musixtex 的工作之一便是计算每行中\ctrlseq{elemskip}的宽度。正确的宽度应该使得每一行都恰好“撑满”行内的所有空间。显然,这个值在每行都不尽相同。

三段系统的第1段是使用\verb|etex|命令(或其他plain \TeX 引擎)编译原文件,生成后缀为\hspace{\xkanjiskip}\verb|.mx1|的文件。该文件由行宽、段落缩进等信息开始,并在之后列出每小节相应的可变宽度、不可变宽度。

第3段的程序是\verb|musixflx|程序,用于计算相应的断行和\ctrlseq{elemskip}在每行的值,使得乐谱间距匀称,并刚好在行末结尾。这个程序是使用\textsc{lua}写出的,拥有跨平台机能。该程序在读取\hspace{\xkanjiskip}\verb|.mx1|文件之后,输出后缀为\hspace{\xkanjiskip}\verb|.mx2|的文件,包括每行的小节数和\ctrlseq{elemskip}的值。

最后,文档需要再次通过\TeX 编译,读取\hspace{\xkanjiskip}\verb|.mx2|的值,并输出优化后的乐谱布局。

\subsection{例子}
以下是三段编译的例子。在进行第1段编译时,\ctrlseq{elemskip}的值尚未确定,所以\musixtex 使用一个默认的值进行输出。在此之后,输出大概会如下所示。

\begin{music}
    \hsize=120mm
    \leftskip=10mm
    \smallmusicsize
    \nostartrule
    \nobarnumbers
    \setclef18
    \setlines10
    \startpiece
    \Notesp\iul0\jnp j\en
    \notes\nuul0\tjn0h\en
    \NOtes\jq g\en\bar
    \notes\Qjuul hjgh\en
    \NOtes\jq j\en\bar
    \Notes\Djul hg\Djul hj\en\bar
    \NOtes\jq{gZ}\en\raggedstoppiece\contpiece
    \Notesp\iul0\jnp g\en
    \notes\nuul0\tjn0h\en
    \Notes\Djul jk\en\bar
    \Notes\Djul hg\en
    \NOtes\jq e\en\bar
    \Notes\Djul gd\Djul eg\en\bar
    \NOtes\jq{cZ}\en\raggedstoppiece\contpiece
    \Notes\iul0\jn g\en
    \notes\nuul0\jn g\tjn0h\en
    \Notes\Djul ge\en\bar
    \Notes\Djul dc\en
    \NOtes\jq d\en\bar
    \Notes\iul0\jn g\en
    \notes\nuul0\jn g\tjn0h\en
    \Notes\Djul ge\en\bar
    \Notes\Djul dc\en
    \NOtes\jq d\en\raggedstoppiece\contpiece
    \Notesp\iul0\jnp g\en
    \notes\nuul0\tjn0h\en
    \Notes\Djul jk\en\bar
    \Notes\Djul hg\en
    \NOtes\jq e\en\bar
    \Notes\Djul gd\Djul eg\en\bar
    \NOtes\jq{cZ}\en
    \Endpiece
\end{music}

\noindent 此时,每行的间距都相等,但是行并不对齐。在经过\verb|musixflx|并重新编译之后,效果则如下所示。

\begin{music}
    \hsize=120mm
    \leftskip=10mm
    \smallmusicsize
    \nostartrule
    \nobarnumbers
    \setclef18
    \setlines10
    \startpiece
    \Notesp\iul0\jnp j\en
    \notes\nuul0\tjn0h\en
    \NOtes\jq g\en\bar
    \notes\Qjuul hjgh\en
    \NOtes\jq j\en\bar
    \Notes\Djul hg\Djul hj\en\bar
    \NOtes\jq{gZ}\en\bar
    \Notesp\iul0\jnp g\en
    \notes\nuul0\tjn0h\en
    \Notes\Djul jk\en\bar
    \Notes\Djul hg\en
    \NOtes\jq e\en\bar
    \Notes\Djul gd\Djul eg\en\bar
    \NOtes\jq{cZ}\en\bar
    \Notes\iul0\jn g\en
    \notes\nuul0\jn g\tjn0h\en
    \Notes\Djul ge\en\bar
    \Notes\Djul dc\en
    \NOtes\jq d\en\bar
    \Notes\iul0\jn g\en
    \notes\nuul0\jn g\tjn0h\en
    \Notes\Djul ge\en\bar
    \Notes\Djul dc\en
    \NOtes\jq d\en\bar
    \Notesp\iul0\jnp g\en
    \notes\nuul0\tjn0h\en
    \Notes\Djul jk\en\bar
    \Notes\Djul hg\en
    \NOtes\jq e\en\bar
    \Notes\Djul gd\Djul eg\en\bar
    \NOtes\jq{cZ}\en
    \Endpiece
\end{music}

\noindent 此时\musixtex 已经决定了相应的行数,每行末尾对齐,但是第1行的音符间距略窄于第2行。本例源代码如下。
\begin{quote}
    \fontsize{10pt}{10pt}\selectfont
    \begin{verbatim}
    \hsize=120mm
    \leftskip=10mm
    \smallmusicsize
    \nostartrule
    \nobarnumbers
    \setclef18
    \setlines10
    \startpiece
    \Notesp\iul0\jnp j\en
    \notes\nuul0\tjn0h\en
    \NOtes\jq g\en\bar
    \notes\Qjuul hjgh\en
    \NOtes\jq j\en\bar
    \Notes\Djul hg\Djul hj\en\bar
    \NOtes\jq{gZ}\en\bar
    \Notesp\iul0\jnp g\en
    \notes\nuul0\tjn0h\en
    \Notes\Djul jk\en\bar
    \Notes\Djul hg\en
    \NOtes\jq e\en\bar
    \Notes\Djul gd\Djul eg\en\bar
    \NOtes\jq{cZ}\en\bar
    \Notes\iul0\jn g\en
    \notes\nuul0\jn g\tjn0h\en
    \Notes\Djul ge\en\bar
    \Notes\Djul dc\en
    \NOtes\jq d\en\bar
    \Notes\iul0\jn g\en
    \notes\nuul0\jn g\tjn0h\en
    \Notes\Djul ge\en\bar
    \Notes\Djul dc\en
    \NOtes\jq d\en\bar
    \Notesp\iul0\jnp g\en
    \notes\nuul0\tjn0h\en
    \Notes\Djul jk\en\bar
    \Notes\Djul hg\en
    \NOtes\jq e\en\bar
    \Notes\Djul gd\Djul eg\en\bar
    \NOtes\jq{cZ}\en\Endpiece
    \end{verbatim}
\end{quote}

\subsection{手动改变布局}
三段编译的一个好处是可以通过少数参数调整乐谱布局,即\ctrlseq{mulooseness}。这个值和\TeX 自身的\ctrlseq{looseness}相当——若这个值为1,程序会将相应段落变得更松散,以在输出时增加1行;若为$-1$则减少1行,以此类推。此时,\textsf{谱表}和\textsf{小节}替代了行和段落。“小节”是任何不含有强制断行的部分乐谱。强制断行的命令包括\ctrlseq{stoppiece}、\ctrlseq{endpiece}、\ctrlseq{zstoppiece}、\ctrlseq{Stoppiece}、\ctrlseq{Endpiece}、\ctrlseq{alaligne}、\ctrlseq{zalaligne}、\ctrlseq{alapage}或\ctrlseq{zalapage}。若以上都不存在,则整段乐谱为一个小节。

在小节中的任意\footnote{为增强可读性,最好在一小节的首、尾。}位置将\ctrlseq{mulooseness}赋为非0值,则\musixtex 会将此小节用非默认的行数排版

例如,将上例最后一行改为
\begin{verbatim}
\NOtes\jq{cZ}\en\mulooseness=1\Endpiece
\end{verbatim}
会产生如下结果。

\begin{music}
    \hsize=120mm
    \leftskip=10mm
    \smallmusicsize
    \nostartrule
    \nobarnumbers
    \setclef18
    \setlines10
    \startpiece
    \Notesp\iul0\jnp j\en
    \notes\nuul0\tjn0h\en
    \NOtes\jq g\en\bar
    \notes\Qjuul hjgh\en
    \NOtes\jq j\en\bar
    \Notes\Djul hg\Djul hj\en\bar
    \NOtes\jq{gZ}\en\bar
    \Notesp\iul0\jnp g\en
    \notes\nuul0\tjn0h\en
    \Notes\Djul jk\en\bar
    \Notes\Djul hg\en
    \NOtes\jq e\en\bar
    \Notes\Djul gd\Djul eg\en\bar
    \NOtes\jq{cZ}\en\bar
    \Notes\iul0\jn g\en
    \notes\nuul0\jn g\tjn0h\en
    \Notes\Djul ge\en\bar
    \Notes\Djul dc\en
    \NOtes\jq d\en\bar
    \Notes\iul0\jn g\en
    \notes\nuul0\jn g\tjn0h\en
    \Notes\Djul ge\en\bar
    \Notes\Djul dc\en
    \NOtes\jq d\en\bar
    \Notesp\iul0\jnp g\en
    \notes\nuul0\tjn0h\en
    \Notes\Djul jk\en\bar
    \Notes\Djul hg\en
    \NOtes\jq e\en\bar
    \Notes\Djul gd\Djul eg\en\bar
    \NOtes\jq{cZ}\en
    \mulooseness=1\Endpiece
\end{music}

\noindent 而与之相反,
\begin{verbatim}
\NOtes\jq{cZ}\en\mulooseness=-1\Endpiece
\end{verbatim}
会产生如下结果。

\begin{music}
    \hsize=120mm
    \leftskip=10mm
    \smallmusicsize
    \nostartrule
    \nobarnumbers
    \setclef18
    \setlines10
    \startpiece
    \Notesp\iul0\jnp j\en
    \notes\nuul0\tjn0h\en
    \NOtes\jq g\en\bar
    \notes\Qjuul hjgh\en
    \NOtes\jq j\en\bar
    \Notes\Djul hg\Djul hj\en\bar
    \NOtes\jq{gZ}\en\bar
    \Notesp\iul0\jnp g\en
    \notes\nuul0\tjn0h\en
    \Notes\Djul jk\en\bar
    \Notes\Djul hg\en
    \NOtes\jq e\en\bar
    \Notes\Djul gd\Djul eg\en\bar
    \NOtes\jq{cZ}\en\bar
    \Notes\iul0\jn g\en
    \notes\nuul0\jn g\tjn0h\en
    \Notes\Djul ge\en\bar
    \Notes\Djul dc\en
    \NOtes\jq d\en\bar
    \Notes\iul0\jn g\en
    \notes\nuul0\jn g\tjn0h\en
    \Notes\Djul ge\en\bar
    \Notes\Djul dc\en
    \NOtes\jq d\en\bar
    \Notesp\iul0\jnp g\en
    \notes\nuul0\tjn0h\en
    \Notes\Djul jk\en\bar
    \Notes\Djul hg\en
    \NOtes\jq e\en\bar
    \Notes\Djul gd\Djul eg\en\bar
    \NOtes\jq{cZ}\en
    \mulooseness=-1\Endpiece
\end{music}

%\noindent which is tighter than you would ever want, but serves to further
%demonstrate the use of \keyindex{mulooseness}.

在手动键入\musixtex 输入文件时(在只使用五线谱时已无太大必要,因为有\textbf{PMX}的存在),可以使用如下步骤。
\begin{enumerate}\itemsep0pt
    \item 逐音符组(\verb|\notes|或相似命令)输入内容,需要使用适当的水平间距命令(见\ref{newspacings}节),使得可变宽度和音符长度有相应的对应关系。这部分内容将在第\ref{preparing}章详细说明。
    \item \TeX~$\Longrightarrow$ {\tt musixflx} $\Longrightarrow$ \TeX.
    \item 观察输出文件,判断是否需要手动调整(例如,排版者可能希望乐谱排满若干页)。
    \item 若需要调整,删除\verb|mx2|文件,调整\ctrlseq{mulooseness},重新进行编译。
\end{enumerate}

另一种方法是定义\ctrlseq{linegoal}的值,但是此时\ctrlseq{mulooseness}必须为0。这两个变量在一小节结束后都会自动清零。

对于长篇乐谱(4页以上),只有1个小节和一个\ctrlseq{mulooseness}值是不明智的,因为此时不仅需要两端对齐,也需要乐谱占整数页。这时,可以通过\ctrlseq{alapage}或\ctrlseq{alaligne}\footnote{\textbf{PMX}也使用这类命令。}。

\musixtex 将伸缩量完全在外部计算,并在排版时已经确定相应的值,这使得符杠、连音线等元素的长度可以严格定义。

\subsection{\texttt{musixflx}的使用}
多数系统上,\verb|musixflx|可以直接通过命令行调用。
\begin{quote}
    \verb|musixflx (文件名).mx1|
\end{quote}

\begin{quote}
    \verb|d | 将调试信息在屏幕上输出\\
    \verb|f | 将调试信息输出到\verb|(文件名).mxl|\\
    \verb|s | 将行的计算结果直接输出到屏幕
\end{quote}
为了简化输入,程序输入的文件名可以带\verb|mx1|、\verb|tex|或不带后缀名;程序在这三种情况下都能正常使用

\subsection{致新手:注意空格!}
因为\musixtex 宏集的特点,打谱时对于空格、空行必须格外小心。新手最常见的错误是在一行中间\textsf{或结尾}多加了一个空格。这类空格(或不通过\musixtex 命令正常输入的其他字符)不会被\musixtex 记录,但是仍然会被\TeX 命令输出。这样的错误会使得乐谱排版时出现\verb|overfull \hbox|一类的错误,并且会影响布局。

避免这类错误的最好方法是不输入不必要的空格,并在不以控制序列结束的行末加上\verb|%|去除换行导致的空格。

另一个需要注意的点是,不要在音乐范围内使用\ctrlseq{hskip}或\ctrlseq{kern},除非在宽度为0的盒子中,如\ctrlseq{rlap}、\verb|\llap|、\verb|\zcharnote|、\verb|\uptext|等。对于可变的宽度,如\ctrlseq{noteskip}、\verb|\elemskip|、\verb|\afterruleskip|、\verb|\beforeruleskip|\footnote{注意,\zctrlseq{hardspace}不在此列,这是\musixtex 宏集特许的加入水平间距的方法。}等,不要进行赋值。


\section{其他}
\subsection{从乐谱中提取声部} 
见第\ref{parts}章。

\subsection{乐谱及音符的大小}
\musixtex 宏集默认的乐谱大小是\SI{20}{pt},但是也可以使用16(简谱使用)、24、\SI{29}{pt}的尺寸。每个乐器也能有自己的乐谱尺寸(一般小于默认尺寸)。此外,音符、符杠等都能通过命令改变尺寸。
\chapter{输入文档的格式}
\section{\musixtex 的基本规定}
\musixtex 的输入文件和一般的\TeX 文件类似。使用\musixtex 宏包的命令是\verb|\input musixtex|;相应地,使用简谱宏包\musixjnp 的命令是\verb|\input musixjnp|。宏集内的其他宏包也如此调用。

\musixtex\enspace 宏集默认乐谱最多有6个声部。如果需要更多的声部,可以使用\textsf{musixadd}\enspace 或\enspace\textsf{musixmad}\enspace 宏包扩展到9或12个。如果还需要更多声部,可以使用\penalty-10000\verb|\setmaxinstruments|、\hfill\verb|\setmaxgroups|、\hfill\verb|\setmaxslurs|、\hfill\verb|\setmaxtrills|、\penalty-10000\verb|\setmaxoctlines|等命令手动修改。


\section{常用的设置命令}\label{whatspecify}

\subsection{全局乐谱尺寸}
\musixtex 设置了4种乐谱尺寸:小(16 pt)、中(20 pt,默认)、大(24 pt)、特大(29 pt)。或需要非默认尺寸,分别使用命令\ctrlseq{smallmusicsize}、\ctrlseq{largemusicsize}、\ctrlseq{Largemusicsize}。在使用该命令时,所有的音乐内容都会相应缩放。

简谱推荐使用“小”尺寸;线简混排时参考第??章的内容。

\subsection{乐器个数}
命令\ctrlseq{instrumentnumber} $n$表示乐谱中乐器的个数,默认为1.

\subsection{谱号}
各谱表的默认谱号是高音谱号。修改谱号的命令为\ctrlseq{setclef}\verb|{|$n$\verb|}{|$s_1s_2s_3s_4\ldots$\verb|}|。$n$是乐器从下往上数的序号,$s_1$、$s_2$等是该乐器从下往上的谱表序号对应的谱号。

简谱应当使用8(没有谱号)。

\subsection{节拍}
所有声部通用的节拍记号可以使用\ctrlseq{generalmeter}\verb|{|$m$\verb|}|命令。$m$是拍号相应的命令,简谱使用\ctrlseq{inijfrac}\verb|{|$a$\verb|}{|$b$\verb|}|(乐谱首)或\ctrlseq{jfrac}\verb|{|$a$\verb|}{|$b$\verb|}|(乐谱中切换拍号),a为上方的数,b为下方的数。

若使某乐器的节拍与全局设置不同,则应该使用
\ctrlseq{setmeter}\verb|{|$n$\verb|}{{|$m_1$\verb|}{|$m_2$\verb|}|\penalty-10000 \verb|{|$m_3$\verb|}{|$m_4$\verb|}}|,与谱表相似。

\subsection{乐器名}
乐器名使用\ctrlseq{setname}\verb|{|$n$\verb|}{|{\fangsong 乐器名}\verb|}|命令设置,程序会自动将乐器名称居中放置在\ctrlseq{parindent}宽度的范围内。

\subsection{乐器组}\label{curlybrackets}
默认情况下,乐器间没有连接。若需要排合唱谱(如上一章中的《喀秋莎》例谱),则需要方括号分组。插入方括号的命令是
\begin{quote}
\ctrlseq{songtop}\verb|{|$n$\verb|}|\\
\ctrlseq{songbottom}\verb|{|$m$\verb|}|
\end{quote}

若需要多个方括号,则使用类似的\ctrlseq{grouptop}和\ctrlseq{groupbottom}命令。

\section{纯简谱谱表常用的设置}
除上述内容以外,简谱谱表还经常需要使用\ctrlseq{nostartrules}(取消每行左侧的连线)和\ctrlseq{sepbarrules}(每个乐器的小节线分开)两条命令。


\endinput



\include{preparing}

\end{document}
这篇文章是我在编写\TeX 使用的简谱宏包时的一些探索和尝试。此宏包是作为Musix\TeX 的延伸,可以用于五线谱和简谱混排。

\section{字体和字号}
\subsection{乐谱字体的设计}
Musix\TeX 使用六种不同大小的字体:10.24\quartspace pt、12.8\quartspace pt、16\quartspace pt、20\quartspace pt、24\quartspace pt、29\quartspace pt。可以看出,这个系列本质上可以近似成以1.2为公比的一个序列。

\begin{table}[!htbp]
\centering\small
\begin{tabular}{cc}
\toprule
\heiti 实际字形(\textbf{pt}) & \heiti 近似字号(\textbf{pt})\\
\midrule
10.24 & 11.57\\
12.8 & 13.89\\
16 & 16.67\\
20 & 20\\
24 & 24\\
29 & 28.8\\
\bottomrule
\end{tabular}
\end{table}

由于我的简谱字形将数字的高设定为0.7\quartspace em,且在歌词以10\quartspace pt显示时数字大概应该以12\quartspace pt高度显示(经验值),因此正常大小的简谱字号为
\[12\ \mathrm{pt}\left(\frac{1\ \mathrm{em}}{0.7\ \mathrm{em}}\right)=17.14\ \mathrm{pt}\approx17.28\ \mathrm{pt}\mbox{。}\]
由于17.28\quartspace pt也在\TeX 最初的字体序列之中,因此可以直接把六种大小不同的字体对应为10\quartspace pt、12\quartspace pt、14.4\quartspace pt、17.28\quartspace pt、20.74\quartspace pt、24.88\quartspace pt。

\subsection{文本字体}
Musix\TeX 在使用的时候定义了7\quartspace pt、8\quartspace pt、9\quartspace pt、10\quartspace pt、10.95\quartspace pt、12\quartspace pt、14.4\quartspace pt、17.28\quartspace pt、20.74\quartspace pt、24.88\quartspace pt等字号,即使文本上也十分适合plain\TeX 的排版。简谱宏包也需要导入相应大小的汉字字体。
