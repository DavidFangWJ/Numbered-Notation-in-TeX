\makeatletter

%11 点 (实际为 10.95 pt) 字体修正
%2018-06-20
\font\elevenrm=cmr10 scaled \magstephalf %原代码字号为 9 × 1.095 = 9.855
\font\elevenbf=cmbx10 scaled \magstephalf
\font\elevenit=cmti10 scaled \magstephalf
\font\elevenbi=cmbxti10 scaled \magstephalf
\font\elevensc=cmcsc10 scaled \magstephalf

%简谱字体 (可与文本字体类比)
%已经改变了大小,不改名字是因为我懒
\font\jpeight=jpfont at 8pt
\font\jpten=jpfont at 10pt
\font\jptwelve=jpfont at 12pt
\font\jpfrt=jpfont
\font\jpsvt=jpfont scaled \magstep1
\font\jptwty=jpfont scaled \magstep2

%重写 MusixTeX 字体调用心尺寸计算 (去除西欧中世纪音乐字体, 因不和简谱混排)
%2018-06-20
\newdimen\jpsiz@
\newdimen\jp@xshift
\newdimen\jp@yshift
\newdimen\ulin@dist

\newif\ifc@ntinu@
\c@ntinu@false

\let\jpnorfont\jpfrt
\let\jpfont\jpnorfont

\def\set@Largenotesize{\let\musixfont\musicLargefont
  \let\jpfont\jpLargefont
  \b@amthick.3456\Interligne \interbeam1.08\Interligne }

\def\set@largenotesize{\let\musixfont\musiclargefont
  \let\jpfont\jplargefont
  \b@amthick.288\Interligne \interbeam.9\Interligne }

\def\set@normalnotesize{\let\musixfont\musicnorfont
  \let\jpfont\jpnorfont
  \b@amthick.24\Interligne \interbeam\p@seven5\Interligne }

\def\set@smallnotesize{\let\musixfont\musicsmallfont
  \let\jpfont\jpsmallfont
  \b@amthick\p@ne92\Interligne \interbeam.6\Interligne }

\def\set@tinynotesize{\let\musixfont\musictinyfont
  \let\jpfont\jptinyfont
  \b@amthick\p@ne536\Interligne \interbeam.48\Interligne }

\def\comput@fonts{%    
  \ifnum\musicsize=\sixt@@n
    \let\musicLargefont\musictwentyfour
    \let\musiclargefont\musictwenty
    \let\musicnorfont\musicsixteen
    \let\musicsmallfont\musicthirteen
    \let\musictinyfont\musiceleven
    \let\jpLargefont\jpsvt
    \let\jplargefont\jpfrt
    \let\jpnorfont\jptwelve
    \let\jpsmallfont\jpten
    \let\jptinyfont\jpeight
    \let\slurd\slurdsixteen \let\sluru\slurusixteen
    \let\hslurd\hslurdsixteen \let\hsluru\hslurusixteen
    \let\meternorfont\tenbf \let\metersmallfont\eightbf  % version 1.16  RDT
    \let\meterbigfont\twelvebf \let\meterlargefont\frtbf
    \let\meterLargefont\svtbf
    \let\tinyppff\ppfftwelve   % version 1.17  RDT
    \let\smallppff\ppfftwelve
    \let\normppff\ppffsixteen
    \let\medppff\ppfftwenty
    \def\txtfont{\ifdim\internote<.95\Internote \tinytype\it  % version 1.22 RDT
    \else\ifdim\internote<1.19\Internote \smalltype\it
    \else\ifdim\internote<1.43\Internote \Smalltype\it
    \else\normtype\it\fi\fi\fi}%
  \else\ifnum\musicsize=\@xxiv
    \let\musicLargefont\musictwentynine
    \let\musiclargefont\musictwentynine
    \let\musicnorfont\musictwentyfour
    \let\musicsmallfont\musictwenty
    \let\musictinyfont\musicsixteen
    \let\jpLargefont\jptwty
    \let\jplargefont\jptwty
    \let\jpnorfont\jpsvt
    \let\jpsmallfont\jpfrt
    \let\jptinyfont\jptwelve
    \let\sluru\slurutwenty \let\slurd\slurdtwenty
    \let\hsluru\hslurutwenty \let\hslurd\hslurdtwenty
    \let\meternorfont\frtbf \let\metersmallfont\twelvebf % version 1.18  RDT
    \let\meterbigfont\svtbf \let\meterlargefont\twtybf
    \let\meterlargefont\twfvbf
    \let\tinyppff\ppffsixteen  
    \let\smallppff\ppfftwenty
    \let\normppff\ppfftwentyfour
    \let\medppff\ppfftwentynine
    \def\txtfont{\ifdim\internote<.95\Internote \smalltype\it % version 1.22 RDT
    \else\ifdim\internote<1.19\Internote \Smalltype\it
    \else\ifdim\internote<1.43\Internote \normtype\it
    \else\medtype\it\fi\fi\fi}%
  \else\ifnum\musicsize=\@xxix
    \let\musicLargefont\musictwentynine
    \let\musiclargefont\musictwentynine
    \let\musicnorfont\musictwentynine
    \let\musicsmallfont\musictwentyfour
    \let\musictinyfont\musictwenty
    \let\jpLargefont\jptwty
    \let\jplargefont\jptwty
    \let\jpnorfont\jptwty
    \let\jpsmallfont\jpsvt
    \let\jptinyfont\jpfrt
    \let\sluru\slurutwenty \let\slurd\slurdtwenty
    \let\hsluru\hslurutwenty \let\hslurd\hslurdtwenty
    \let\slurud\slurutwentyd \let\slurdd\slurdtwentyd %    +ickd
    \let\hslurud\hslurutwentyd \let\hslurdd\hslurdtwentyd %+ickd
    \let\meternorfont\svtbf \let\metersmallfont\frtbf     % version 1.18  RDT
    \let\meterbigfont\twtybf \let\meterlargefont\twfvbf   % version 1.24 typos corrected
    \let\meterLargefont\twfvbf
    \let\tinyppff\ppfftwenty   
    \let\smallppff\ppfftwentyfour
    \let\normppff\ppfftwentynine
    \let\medppff\ppfftwentynine
    \def\txtfont{\ifdim\internote<.95\Internote \normtype\it  % version 1.22 RDT
    \else\ifdim\internote<1.19\Internote \medtype\it
    \else\ifdim\internote<1.43\Internote \bigfont\it
    \else\Bigfont\it\fi\fi\fi}%
\else
    \ifnum\musicsize=\tw@nty
    \else\ifnum\musicsize=\z@
         \else\errmessage{\noexpand\musicsize=\the\musicsize\space not supported,
           set to default of 20}%
         \fi
    \fi\musicsize\tw@nty
    \let\musicLargefont\musictwentynine
    \let\musiclargefont\musictwentyfour
    \let\musicnorfont\musictwenty
    \let\musicsmallfont\musicsixteen
    \let\musictinyfont\musicthirteen
    \let\jplargefont\jpsvt
    \let\jpLargefont\jptwty
    \let\jpnorfont\jpfrt
    \let\jpsmallfont\jptwelve
    \let\jptinyfont\jpten
    \let\sluru\slurutwenty \let\slurd\slurdtwenty
    \let\hsluru\hslurutwenty \let\hslurd\hslurdtwenty
    \let\meternorfont\twelvebf \let\metersmallfont\tenbf   % version 1.18  RDT
    \let\meterbigfont\frtbf \let\meterlargefont\svtbf
    \let\meterLargefont\twtybf
    \let\tinyppff\ppfftwelve  
    \let\smallppff\ppffsixteen
    \let\normppff\ppfftwenty
    \let\medppff\ppfftwentyfour
    \def\txtfont{\ifdim\internote<.95\Internote \smalltype\it
    \else\ifdim\internote<1.19\Internote \Smalltype\it
    \else\ifdim\internote<1.43\Internote \normtype\it
    \else\medtype\it\fi\fi\fi}%
\fi\fi\fi}

\def\comput@sizes{%
  \Interligne\fontdimen\fiv@\musicnorfont
  \Internote\h@lf\Interligne \big@spc.6\Interligne
  \qn@width\fontdimen\si@\musixfont
  \wn@width1\qu@rt\qn@width
  \txt@ff\h@lf\qn@width
  \qd@skip\qn@width\advance\qd@skip-\hlthick
  % 新增部分
  \jpsiz@\tw@\fontdimen\si@\jpfont
  \jp@xshift\fontdimen\si@\jpfont
  \advance\jp@xshift-\fontdimen\si@\musicnorfont
  \divide\jp@xshift\tw@
  \jp@yshift\tw@\fontdimen\fiv@\musixfont
  \advance\jp@yshift-\p@seven\fontdimen\fiv@\jpfont
  \ulin@dist.2\fontdimen\fiv@\jpfont}

%修正 0 线 (简谱用) 的情况
\def\C@Inter{%  RDT: corrected to work if \nblines > 6   (version 1.23)
  \stem@skip\interportee
  \ifnum\nblines=\thr@@
    \advance\stem@skip-\@ight\internote
  \else\ifnum\nblines=\z@
    \advance\stem@skip-\@ight\internote
  \else
    \advance\stem@skip-\nblines\internote
    \advance\stem@skip-\nblines\internote
    \advance\stem@skip\tw@\internote
  \fi\fi}

\def\jpchar{\jpfont\char}

\def\jxgetn@i#1\relax{\n@viii\z@ \n@i\maxdimen \n@ii\z@% \n@iii\z@%最大宽度
  \edef\t@ruc{\expandafter\f@tok #1\relax\af@tok}%
  \edef\s@uite{\s@tok #1\empty\af@tok}%
%% pas lettre
  \ifcat a\t@ruc \n@i\expandafter`\t@ruc\relax
%%先把 \n@i 变成“距离中音 1 的值”
    \ifnum\n@i=82 \n@i48%大写字母 R 表示休止
	\else\ifnum\n@i=88
	\else\ifnum\n@i=90 \n@i45
	  \else\ifnum\n@i>96\advance\n@i-99%小写字母, c 是中音 1
	  \else\advance\n@i-81%让 C 对应 -14
	  \fi
	  %低音区, 往下递减
      \loop \ifnum\n@i<\z@ \advance\n@i\s@v@n \advance\n@ii\m@ne
      \repeat
	  %高音区, 往上递增
      \loop \ifnum\n@i>\si@ \advance\n@i-\s@v@n \advance\n@ii\@ne
      \repeat
	  \advance\n@i49%\n@i 的 0 表示音符“1”, 即编码49
	  \ifnum\n@ii<\z@ \multiply\n@ii-1 \advance\n@ii\thr@@%低音点的码位
	\fi\fi\fi\fi
    \edef\ss@uite{\noexpand\n@fon{\s@uite}}%
%%此处暂时没有改写, 一定有 bug, 不要用
  \else
    \ifnum\n@i=45 \edef\ss@uite{\noexpand\n@fon{\s@uite}}%
	\else
      \let\ss@uite\empty
      \let\alt@suite\empty
    %\ifcat 1\t@ruc
    %  \if =\t@ruc \let\@TI\na  \C@GET \fi
    %  \if *\t@ruc \sk \C@Get \fi
    %  \if .\t@ruc \let\@TI\pt   \C@GET \fi
    %  \if >\t@ruc \let\@TI\dsh \C@GET \fi
    %  \if <\t@ruc \let\@TI\dfl \C@GET \fi
    %  \if !\t@ruc \transpose\normaltranspose \C@Get \fi  
    %  \if '\t@ruc \advance\transpose\s@v@n   \C@Get \fi  
    %  \if `\t@ruc \advance\transpose-\s@v@n  \C@Get \fi
    %  \ifnum\n@viii<\maxdimen \n@i#1\fi
    %\else
    %  \if ^\t@ruc \let\@TI\sh \C@GET \fi
    %  \if _\t@ruc \let\@TI\fl \C@GET \fi
    %\fi\alt@suite
  \fi\fi}

\def\sadv@box#1{\hbox\@to\noteskip{\kernm\jp@xshift#1\hss}\advance\locx@skip\noteskip}
\def\s@hbox#1{\hbox\@to\noteskip{\kernm\jp@xshift#1\kern\jp@xshift\hss}}

%%画单条减时线
\def\dr@wsingulin@{\y@ii\n@vii\ulin@dist%这个数是正数, 因此以向下为正
  \advance\y@ii-\jp@yshift
  \y@iii\y@ii
  \advance\y@ii-\hlthick%往上
  \advance\y@iii\hlthick%往下
  \y@iv\h@lf\jpsiz@
  \ifcase\n@pt \y@iv\h@lf\jpsiz@ \or \y@iv\jpsiz@ \or \y@iv\@ne\h@lf\jpsiz@ \fi
  \ifc@ntinu@ \y@iv\noteskip \fi
  \vrule\@height-\y@ii\@depth\y@iii\@width\y@iv
  \kernm\y@iv}

%%画减时线及上下加点
\def\dr@wadd{\y@v\jp@yshift
  \ifnum\n@ii>\thr@@ \advance\y@v-\n@vii\ulin@dist \fi% 是低音, 可能需要向下
  \ifnum\n@ii>\z@ \raise\y@v\hbox{\jpchar\n@ii}\fi
  \ifx\n@wline\relax
    \loop\ifnum\n@vii>\z@
      \dr@wsingulin@ \advance\n@vii\m@ne
    \repeat
  \else \c@ntinu@false
    \loop\ifnum\n@vii>\z@
      \dr@wsingulin@ \advance\n@vii\m@ne
	  \ifnum\n@vii=\n@wline \c@ntinu@true \let\n@line\n@wline \fi
    \repeat \let\n@wline\relax
  \fi}

\def\jgetn@i{\jxgetn@i}
%音符书写
\def\jw@n{\kernm\jp@xshift
  \ifnum\n@i<\@c
    \n@vii\n@line
%原本这块用于加线, 现在改为减时线
    \dr@wadd
	\raise\jp@yshift\adv@box\n@sym% \else \raise\ulin@dist\hbox{\n@sym}\fi
    \ss@uite% \advancetrue
  \fi \kern\jp@xshift}
%分析输入
\def\jg@n#1{\check@staff
  \jgetn@i#1\relax \let\n@fon\jg@n \let\n@sym\j@n \jw@n}
\def\jg@np#1{\check@staff
  \jgetn@i#1\relax \let\n@fon\jg@np \let\n@sym\j@np \jw@n}
\def\jg@npp#1{\check@staff
  \jgetn@i#1\relax \let\n@fon\jg@npp \let\n@sym\j@npp \jw@n}

\def\jfrac#1#2{\setbox\toks@box\vbox{\hbox{\ \meterfont#1}%
  \hbox{\ \meterfont #2}}\kernm\h@lf\jpsiz@\vbox\@to\@ight\internote{\offinterlineskip
  \vss\hbox\@to\wd\toks@box{\hss\meterfont#1\hss}\vss
  \vbox\@to0.4pt{\hrule\@width\wd\toks@box}\vss
  \hbox\@to\wd\toks@box{\hss\meterfont#2\hss}\vss}\kern\h@lf\jpsiz@}

\def\j@n{\jpchar\n@i}
\def\j@np{\jpchar\n@i.}
\def\j@npp{\let\n@pt\tw@ \jpchar\n@i\char1}

%正式输入的命令
\def\jq{\let\n@line\z@ \let\n@pt\z@ \jg@n}
\def\jc{\let\n@line\@ne \let\n@pt\z@ \jg@n}
\def\jcc{\let\n@line\tw@ \let\n@pt\z@ \jg@n}
\def\jccc{\let\n@line\thr@@ \let\n@pt\z@ \jg@n}
\def\jcccc{\let\n@line\f@ur \let\n@pt\z@ \jg@n}

\def\jqp{\let\n@line\z@ \let\n@pt\@ne \jg@np}
\def\jcp{\let\n@line\@ne \let\n@pt\@ne \jg@np}
\def\jccp{\let\n@line\tw@ \let\n@pt\@ne \jg@np}
\def\jcccp{\let\n@line\thr@@ \let\n@pt\@ne \jg@np}
\def\jccccp{\let\n@line\f@ur \let\n@pt\@ne \jg@np}

\def\jqpp{\let\n@line\z@ \let\n@pt\tw@ \jg@npp}
\def\jcpp{\let\n@line\@ne \let\n@pt\tw@ \jg@npp}
\def\jccpp{\let\n@line\tw@ \let\n@pt\tw@ \jg@npp}
\def\jcccpp{\let\n@line\thr@@ \let\n@pt\tw@ \jg@npp}
\def\jccccpp{\let\n@line\f@ur \let\n@pt\tw@ \jg@npp}

%用于连接下划线的情况
\def\iul{\c@ntinu@true \let\n@line\@ne}
\def\iuul{\c@ntinu@true \let\n@line\tw@}
\def\iuuul{\c@ntinu@true \let\n@line\thr@@}
\def\iuuuul{\c@ntinu@true \let\n@line\f@ur}

\def\tul{\let\n@wline\relax \c@ntinu@false}
\def\tuul{\let\n@line\tw@ \let\n@wline\@ne}
\def\tuuul{\let\n@line\thr@@ \let\n@wline\tw@}
\def\tuuuul{\let\n@line\f@ur \let\n@wline\thr@@}

\def\jn{\let\n@pt\z@ \jg@n}
\def\jnp{\let\n@pt\@ne \jg@np}
\def\jnpp{\let\n@pt\tw@ \jg@npp}

%简化输入
\def\tjn{\tul\jn}
\def\tjnp{\tul\jnp}
\def\tjnpp{\tul\jnpp}

\def\Djul#1#2{\iul\jn#1\tjn#2}
\def\Djuul#1#2{\iuul\jn#1\tjn#2}

%简谱调号
\def\jpkey#1#2{{\lyricsoff\hbox\@to.5em{\jpten #1\hss} = #2\lyricson}}

\let\n@wline\relax
\makeatother
\endinput